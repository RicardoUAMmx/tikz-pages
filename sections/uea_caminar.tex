% -----------------------------------------------------------------------------
% Start File: sections/uea_caminar.tex
% Author: Ricardo Hernández Osorio
% Created: Tuesday, May-23-2023
% Description: Contents UEA Ensayo
% -----------------------------------------------------------------------------

Me resulta difícil hablar sobre el tema que se ha vivido en toda la
\textbf{Universidad Autónoma Metropolitana}, no solo por que me es difícil
entender a ambas partes del conflicto, sino por que el tema es tan grande,
amplio y de tanto tiempo atrás que no me es sencillo establecer un comienzo; aún
en un ensayo que pretende responder puntualmente a cuestiones que ya de por sí
parecen ser arbitrarias y sin sentido, y por lo tanto las plantearé como
\textbf{cuestiones a trabajar y no como cuestiones a resolver}, ya sea como una
crítica a su planteamiento o bien analizando sus conflictos que se desprenden de
aquellas. Y es que vale la pena preguntar sobre la arbitrariedad y su imposición
como una forma de hacer pensar aquello que nos es imposible transmitir de una
manera más inteligente lo no pensado, o lo que es lo mismo: \textbf{lo que es
por que sí y así tiene que ser}. Tal como como se ha visto en escuelas del siglo
pasado, castigando y repitiendo, llenando planas y planas de palabras sin
sentido salvo salir del conflicto; cumpliendo por un lado y haciendo cumplir por
el otro. Mientras uno cree que repetir lo hace pensante el otro piensa que
repetir es pensar. Y así constituimos nuestra identidad universitaria con
castigos y castigados, amos y esclavos de su condición \textbf{Maestro - 
Alumno}.

Primeramente para responder la cuestión \emph{¿Con qué nos quedamos?} tendría
que habérsenos dado algo y de ese algo discriminar sus partes en función de
valores individuales o colectivos. Pero ¿Es cierto que necesariamente todos los
integrantes de la comunidad se les dio algo con el paro? Es decir, lo anterior
de manera implícita nos dice que \emph{no existen integrantes indiferentes}, lo
cual ya es cuestionable, e incluso nos lleva a preguntarnos sobre lo que podría
ser revelador: \textbf{¿Cuál es la razón de la indiferencia de aquellos
individuos?} pues podríamos encontrarnos con respuestas como
\textbf{la desgana, la desmotivación, el miedo, la apatía, el servilismo, la
corrupción, la individualidad}; que no solo perpetúan el conflicto que nos
orilló sino que lo promueven en muchos casos. De un modo u otro lo dado puede
ser lo acontecido: \textbf{Violencia} principalmente, ya sea como causa o
también como efecto.

Con un planteamiento así no resultaría raro ni debería de causar sorpresa a
nadie el desprestigio y la falta de credibilidad moderna en las instituciones
universitarias de hoy, pues el ambiente ambiguo y la apariencia por encima de la
esencia resulta la norma. Es decir, conformamos aquella institución que promueve
educación profesional y altos valores sociales pero que ejerce desde su corazón,
llámese alumnos o profesores, burocracia como solución, miedo e intimidación
como valor, repetición como educación y democracia como instrumento de
consolidación social por sobre las minorias y la diferencias.

Por tanto, preguntar \emph{¿Con qué nos quedamos después del paro?} no solo
refleja antipatía con el movimiento sino ignorancia con sus temáticas y
exigencias, ajustando la perspectiva del otro en el yo, quedandonos así con un
\textbf{otro tolerado y un otro intolerable}, digno de un sociopata que acepta
sin conceder ni entender del todo, pero \textbf{politicamente correcto}. Un
sujeto sujetado a su condición social intentando ser percibido empático con su
comunidad.

De lo anterior podemos ya deducir que no nos es posible \emph{quedarnos} con una
parte del movimiento sin deformar su significado. Tampoco no es posible
\emph{quedarnos} con una parte no habiendo tomado algo, en este caso, sin haber
entendido todo el movimiento y lo acontecido. Tampoco nos es posible apropiarnos
de significado siendo indiferentes, pues tal indiferencia es también una
posición a favor o en contra, ya sea violentando con nuestra individualidad o
siendo víctimas del miedo. En todo caso, \textbf{la acción nos hará libres}.

Sartre ya nos demostraba lo inaccesible que puede ser el otro y aún con eso
preguntarnos por él y precisamente nuestra persistencia sobre nuestras
imposibilidades nos distingue y hemos sabido sobresalir como especie, y en
aquello radica mi \textbf{significado personal} sobre los acontecimientos
durante el paro: \textbf{Siendo imperfectos en un universo físicamente perfecto,
finitos en medio de la infinitud, inaccesibles para otros; y aún con ello
revelarse contra lo que intentan hacer de uno mismo}. La violencia contraria a
la consciencia y el pensamiento, nos arrebata nuestra existencia, siendo tomada
por nuestras vulnerabilidades y haciendo de nosotros ellos, productivos
disfuncionales arrojados al mundo \textbf{haciendo lo que han hecho de nosotros,
ya sea por acción o por reacción, eligiendo ser lo que hemos de decidir ser}.
Nada de lo posible en una sociedad así parece destacable ni tampoco con signos
de mejoría, al contrario, perecer parece su destino y nada más alejado de la
realidad pues en tiempos violentos el cine apocalíptico, retro y romantizando el
ayer como un mundo mejor es la moda.

Nihilismo, individualismo, inmediatez e impropiedad no parece hacernos
\textbf{mejores como comunidad} pero finalmente lo hecho, hecho; y lo dicho,
dicho. \emph{Nada importa y toda acción en contra no tiene sentido, todo será
igual y cada uno de nosotros será mejor si nos graduamos a tiempo y lo más
pronto posible pues a nuestra edad nuestros padres ya tendrían todo por cuanto
soñaron y nosotros no parece ser que vayamos a ser dueños siquiera de un modesto
auto usado}. ¿Te suena ese pensamiento? Nihilista, individual, inmediato y sin
propiedad no es pensamiento de muchos, sino su realidad, sujetos sujetados
haciendo lo que han hecho de ellos.

Despierta, \textbf{la acción nos hará libres}.

% -----------------------------------------------------------------------------
% End File: sections/uea_caminar.tex
% -----------------------------------------------------------------------------
